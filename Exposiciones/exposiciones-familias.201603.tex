\documentclass[letterpaper,10pt]{article}
\usepackage[utf8x]{inputenc}
\usepackage[margin=1cm,top=.5cm,bottom=.5cm]{geometry}

%opening
\title{BINGE-61 Microcontroladores\\ Exposiciones sobre familias y marcas de $\mu$P}
\author{Ing. Jorge Rivera}
\date{III Cuatrimestre, 2016}

\markboth{}{}

\begin{document}

\maketitle

\section{Descripción}

Los siguientes son los temas asignados para las exposiciones del curso BINGE-61 Microcontroladores. Las exposiciones se realizarán en la clase del día martes 22 de setiembre, 2015. Cada estudiante dispondrá de 10 minutos para su presentación.

\section{Evaluación}

Se evaluará la presentación de la siguiente manera:

\begin{center}
	\begin{tabular}{|l|c|}
	\hline
	Rubro							& Porcentaje \\ \hline\hline
	Datos solicitados 				& 40\% \\ \hline
	Calidad del material presentado & 20\% \\ \hline
	Manejo de la información		& 20\% \\ \hline
	Orden y calidad de la exposición & 20\% \\ \hline\hline
	Total							& 100\% \\ \hline
\end{tabular} 
\end{center}


\section{Exposiciones}

\begin{center}
\begin{tabular}{|c|c|p{4cm}|}\hline
\#  & Familia 					& Encargado			\\ \hline\hline
1   & Xilinx PicoBlaze      	& 					\\\hline
2   & Intel Quark SoC X1000		& 					\\\hline
3   & Atmel SAM3X		 		& 					\\\hline
4   & Broadcom BCM2835/BCM2836	& 					\\\hline
5   & Freescale i.MX6 			& 					\\\hline

6   & Allwinner A20				& 					\\\hline
7   & Texas Instruments AM335x	& 					\\\hline
8   & Microchip PIC32MX			& 					\\\hline
9   & Samsung Exynos 5			& 					\\\hline
10  & Rockchip RK32xx			&					\\\hline

\end{tabular}
\end{center}

\section{Datos solicitados}
\begin{small}
\begin{itemize}
\item Breve reseña del fabricante
\item Características técnicas comunes para la familia
\item Rango de dispositivos en la familia
\item \textbf{Tarjetas de desarrollo disponibles}
\item Lenguajes de programación, sistemas operativos que corren en la tarjeta
\item Costo de las plataformas de desarrollo
\item ¿Porqué deberíamos usar esta plataforma?
\item ¿Porqué NO deberíamos usarla?
\end{itemize}
\end{small}

\end{document}
